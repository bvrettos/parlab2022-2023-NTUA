\documentclass[letterpaper,12pt]{article}

\usepackage{listings}
\usepackage{color}
\usepackage{float}
\usepackage{graphicx}
\usepackage{subcaption}
\usepackage{subfiles}
\usepackage{diagbox}

\definecolor{dkgreen}{rgb}{0,0.6,0}
\definecolor{gray}{rgb}{0.5,0.5,0.5}
\definecolor{mauve}{rgb}{0.58,0,0.82}

\lstset{frame=tb,
  language=C,
  aboveskip=3mm,
  belowskip=3mm,
  showstringspaces=false,
  columns=flexible,
  basicstyle={\small\ttfamily},
  numbers=none,
  numberstyle=\tiny\color{gray},
  keywordstyle=\color{blue},
  commentstyle=\color{dkgreen},
  stringstyle=\color{mauve},
  breaklines=true,
  breakatwhitespace=true,
  tabsize=3
}

\usepackage{tabularx} % extra features for tabular environment
\usepackage{amsmath}  % improve math presentation
\usepackage{graphicx} % takes care of graphic including machinery
\usepackage[margin=1in,letterpaper]{geometry} % decreases margins
\usepackage{cite} % takes care of citations
\usepackage[final]{hyperref} % adds hyper links inside the generated pdf file
\hypersetup{
	colorlinks=true,       % false: boxed links; true: colored links
	linkcolor=blue,        % color of internal links
	citecolor=blue,        % color of links to bibliography
	filecolor=magenta,     % color of file links
	urlcolor=blue         
}
%\documentclass{article}
\usepackage[utf8]{inputenc}
\usepackage[greek,english]{babel}
\usepackage{alphabeta}


\graphicspath{{../Lab1/plot/}{../Lab2/plots}{../Lab3/plots}{../Lab4/plots}{../Lab5/plots}}
\begin{document}

% O Tsafos pe8ane :( Gia panta VhtaPeis
\title{Συστήματα Παράλληλης Επεξεργασίας}
\author{Ομάδα 6 \\
Βασίλειος Βρεττός - el18126, \\
Ανδρέας Βατίστας - el18020 \\}
\date{Σχολή Ηλεκτρολόγων Μηχανικών και Μηχανικών Υπολογιστών, Εθνικό Μετσόβιο Πολυτεχνείο}
\maketitle

% Lab 1
\section{Άσκηση 1 - Εξοικείωση με το περιβάλλον προγραμματισμού}
\subfile{lab_reports/lab1.tex}

% Lab 2
\section{Άσκηση 2 - Παραλληλοποίηση και βελτιστοποίηση αλγορίθμων σε αρχιτεκτονικές κοινής μνήμης}
\subfile{lab_reports/lab2.tex}

% Lab 3
\section{Άσκηση 3 - Αμοιβαίος Αποκλεισμός/Κλειδώματα}
\subfile{lab_reports/lab3.tex}

% Lab 4
\section{Άσκηση 4 - Ταυτόχρονες Δομές Δεδομένων}
\subfile{lab_reports/lab4.tex}

% Lab 5
\section{Άσκηση 5 - Παραλληλοποίηση και βελτιστοποίηση αλγορίθμων σε επεξεργαστές γραφικών}
\subfile{lab_reports/lab5.tex}

\end{document}