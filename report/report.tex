\documentclass[letterpaper,12pt]{article}
\usepackage{tabularx} % extra features for tabular environment
\usepackage{amsmath}  % improve math presentation
\usepackage{graphicx} % takes care of graphic including machinery
\usepackage[margin=1in,letterpaper]{geometry} % decreases margins
\usepackage{cite} % takes care of citations
\usepackage[final]{hyperref} % adds hyper links inside the generated pdf file
\hypersetup{
	colorlinks=true,       % false: boxed links; true: colored links
	linkcolor=blue,        % color of internal links
	citecolor=blue,        % color of links to bibliography
	filecolor=magenta,     % color of file links
	urlcolor=blue         
}
%\documentclass{article}
\usepackage[utf8]{inputenc}
\usepackage[greek,english]{babel}
\usepackage{alphabeta}

\begin{document}

\title{Συστήματα Παράλληλης Επεξεργασίας}
\author{Ομάδα 6 \\
Βασίλειος Βρεττός - el18126, \\
Ανδρέας Βατίστας - el18020, \\
Αλέξανδρος Τσάφος - el18211\\}
\date{Σχολή Ηλεκτρολόγων Μηχανικών και Μηχανικών Υπολογιστών, Εθνικό Μετσόβιο Πολυτεχνείο}
\maketitle




\section{Άσκηση 1}
Η συγκεκριμένη άσκηση είναι εισαγωγική με σκοπό την εξοικείωση μας με τα μηχανήματα του cslab στα οποία εκτελούμε τις εργασίες. Για να δοκιμάσουμε την κατανόησή μας, ζητήθηκε να παραλληλοποιήσουμε το Conway’s Game of Life, ένα απλό παιχνίδι το οποίο “παίζεται” πάνω σε ένα ταμπλό (στην δική μας περίπτωση, διαστάσεων N x N). Το παιχνίδι τρέχει για Κ γύρους (χρονικά διαστήματα). Το κάθε σημείο του ταμπλό έχει 2 καταστάσεις, είτε ζωντανό ή νεκρό. Οι 2 κύριοι κανόνες του παιχνιδιού είναι: \newline

1.	Αν ένα ζωντανό σημείο έχει περισσότερους από 3 γείτονες ή λιγότερους από 2, τότε γίνεται νεκρό στον επόμενο γύρο. Αν έχει ακριβώς 2 ή 3, παραμένει ζωντανό. 

2.	Αν ένα νεκρό σημείο έχει ακριβώς 3 γείτονες, γίνεται ζωντανό στον επόμενο γύρο \newline

Τρέχουμε το παραπάνω σενάριο για 1000 γενιές (γύρους) σε πίνακες διαστάσεων 64x64, 1024x1024 και 4096x4096.

Ο κώδικας ο οποίος ζητείται να παραλληλοποιηθεί είναι ο εξής:



\end{document}