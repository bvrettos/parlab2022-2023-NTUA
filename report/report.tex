\documentclass[letterpaper,12pt]{article}


\usepackage{listings}
\usepackage{color}

\definecolor{dkgreen}{rgb}{0,0.6,0}
\definecolor{gray}{rgb}{0.5,0.5,0.5}
\definecolor{mauve}{rgb}{0.58,0,0.82}

\lstset{frame=tb,
  language=C,
  aboveskip=3mm,
  belowskip=3mm,
  showstringspaces=false,
  columns=flexible,
  basicstyle={\small\ttfamily},
  numbers=none,
  numberstyle=\tiny\color{gray},
  keywordstyle=\color{blue},
  commentstyle=\color{dkgreen},
  stringstyle=\color{mauve},
  breaklines=true,
  breakatwhitespace=true,
  tabsize=3
}


\usepackage{tabularx} % extra features for tabular environment
\usepackage{amsmath}  % improve math presentation
\usepackage{graphicx} % takes care of graphic including machinery
\usepackage[margin=1in,letterpaper]{geometry} % decreases margins
\usepackage{cite} % takes care of citations
\usepackage[final]{hyperref} % adds hyper links inside the generated pdf file
\hypersetup{
	colorlinks=true,       % false: boxed links; true: colored links
	linkcolor=blue,        % color of internal links
	citecolor=blue,        % color of links to bibliography
	filecolor=magenta,     % color of file links
	urlcolor=blue         
}
%\documentclass{article}
\usepackage[utf8]{inputenc}
\usepackage[greek,english]{babel}
\usepackage{alphabeta}

\begin{document}

\title{Συστήματα Παράλληλης Επεξεργασίας}
\author{Ομάδα 6 \\
Βασίλειος Βρεττός - el18126, \\
Ανδρέας Βατίστας - el18020, \\
Αλέξανδρος Τσάφος - el18211\\}
\date{Σχολή Ηλεκτρολόγων Μηχανικών και Μηχανικών Υπολογιστών, Εθνικό Μετσόβιο Πολυτεχνείο}
\maketitle




\section{Άσκηση 1 - Εξοικείωση με το περιβάλλον προγραμματισμού}
Η συγκεκριμένη άσκηση είναι εισαγωγική με σκοπό την εξοικείωση μας με τα μηχανήματα του cslab στα οποία εκτελούμε τις εργασίες. Για να δοκιμάσουμε την κατανόησή μας, ζητήθηκε να παραλληλοποιήσουμε το Conway’s Game of Life, ένα απλό παιχνίδι το οποίο “παίζεται” πάνω σε ένα ταμπλό (στην δική μας περίπτωση, διαστάσεων N x N). Το παιχνίδι τρέχει για Κ γύρους (χρονικά διαστήματα). Το κάθε σημείο του ταμπλό έχει 2 καταστάσεις, είτε ζωντανό ή νεκρό. Οι 2 κύριοι κανόνες του παιχνιδιού είναι: \newline

1.	Αν ένα ζωντανό σημείο έχει περισσότερους από 3 γείτονες ή λιγότερους από 2, τότε γίνεται νεκρό στον επόμενο γύρο. Αν έχει ακριβώς 2 ή 3, παραμένει ζωντανό. 

2.	Αν ένα νεκρό σημείο έχει ακριβώς 3 γείτονες, γίνεται ζωντανό στον επόμενο γύρο \newline

Τρέχουμε το παραπάνω σενάριο για 1000 γενιές (γύρους) σε πίνακες διαστάσεων 64x64, 1024x1024 και 4096x4096. \newline

Ο κώδικας ο οποίος ζητείται να παραλληλοποιηθεί είναι ο εξής.


\begin{lstlisting}
for ( t = 0 ; t < T ; t++ ) {
        #pragma omp parallel for shared (previous, current) private (i,j,nbrs)
         for ( i = 1 ; i < N-1 ; i++ )
            for ( j = 1 ; j < N-1 ; j++ ) {
                nbrs = previous[i+1][j+1] + previous[i+1][j] + previous[i+1][j-1] \
                       + previous[i][j-1] + previous[i][j+1] \
                       + previous[i-1][j-1] + previous[i-1][j] + previous[i-1][j+1];
                if ( nbrs == 3 || ( previous[i][j]+nbrs ==3 ) )
                    current[i][j]=1;
                else
                    current[i][j]=0;
            }

        #ifdef OUTPUT
        print_to_pgm(current, N, t+1);
        #endif
        //Swap current array with previous array
        swap=current;
        current=previous;
        previous=swap;
    }
\end{lstlisting}



Το πρώτο for-loop, το οποίο είναι η αλλαγή των γενιών, δεν έχει νόημα να παραλληλοποιηθεί αφού χρειαζόμαστε την τιμή της προηγούμενης γενιάς για να υπολογίσουμε τις καταστάσεις των κελιών για την επόμενη γενιά. \newline

Η δική μας προσθήκη στον κώδικα είναι η γραμμή.

\begin{lstlisting}
 #pragma omp parallel for shared (previous, current) private (i,j,nbrs)
\end{lstlisting}


Το συγκεκριμένο compiler directive (pragma) είναι χαρακτηριστικό του OpenMP. Είναι μια “οδηγία” προς τον μεταγλωττιστή η οποία του ζητάει να παραλληλοποιηθούν τα εσωτερικά for-loops με νήματα ορισμένα από ένα environmental variable. Στο συγκεκριμένο directive, επίσης ζητάμε:

1.	Οι δείκτες previous, current να μοιράζονται μεταξύ των νημάτων. Η διάσταση Ν μοιράζεται by default.

2.	Οι μεταβλητές i,j,nbrs να είναι ξεχωριστές για κάθε νήμα.
\newline

Την απόδοση του πολυνηματισμού θα εξετάσουμε με χρήση την μετρική του χρόνου, και κατά συνέπεια, του speedup $S=\frac{T_s}{T_p}$ . \newline


\paragraph{Χρόνος Εκτέλεσης}
\paragraph{Speedup}

\paragraph{Συμπεράσματα} \hfill \break
\underline{Για διαστάσεις 64x64}:

Παρατηρούμε ότι η μείωση του χρόνου δεν είναι ακριβώς ανάλογη των αριθμών των νημάτων. Από 1 σε 2 νήματα ή από 2 σε 4 νήματα βλέπουμε σημαντική βελτίωση της απόδοσης. Η εναλλαγή από 4 σε 6 νήματα προσφέρει πολύ μικρότερη αύξηση απόδοσης. Τέλος, από 6 σε 8 νήματα παρατηρούμε \textbf{ΑΥΞΗΣΗ} του χρόνου εκτέλεσης (μηδαμινή). 

Η συγκεκριμένη αύξηση οφείλεται στο overhead που υπάρχει με την χρήση πολυνηματισμού σε μια διεργασία (γέννηση νημάτων, επικοινωνία κ.λ.π.). Ο συγκεκριμένος πίνακας είναι τόσο μικρός που η παραλληλοποίησή του περαιτέρω δεν βγάζει νόημα. Αν είχαμε την δυνατότητα να εξετάσουμε μεγαλύτερο αριθμό νημάτων, πολύ πιθανό να βλέπαμε περαιτέρω αύξηση του χρόνου εκτέλεσης.\newline
\underline{Για διαστάσεις 1024x1024}:

Παρατηρούμε ότι το speedup τείνει να είναι πλήρως γραμμικό. Ο χρόνος υποδιπλασιάζεται με κάθε διπλασιασμό νημάτων. Το συγκεκριμένο ταμπλό φαίνεται ιδανικό για παραλληλοποίηση. Η επικοινωνία μεταξύ νημάτων δεν περιορίζει την απόδοσή τους.




\section{Άσκηση 2 - Παραλληλοποίηση και βελτιστοποίηση αλγορίθμων σε αρχιτεκτονικές κοινής μνήμης}

Στην συγκεκριμένη άσκηση, σκοπός είναι η παραλληλοποίηση 2 διαφορετικών εκδόσεων των αλγορίθμων K-means και Floyd-Warshall. \



\end{document}